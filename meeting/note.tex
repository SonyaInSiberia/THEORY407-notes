\LoadClass[12pt,letterpaper]{article}
\RequirePackage{palatino}
\RequirePackage{verbatim}
\RequirePackage{amsmath,amsfonts,amsthm,amssymb,multirow,xcolor}
\RequirePackage{geometry}
\RequirePackage{graphicx}
\RequirePackage[useregional]{datetime2}
\RequirePackage{hyperref}
%===================================================
\renewcommand{\maketitle}{
\hrule height 1pt
\begin{center}
{\bf \large Meeting Notes}\\[2mm]
THEORY407, Fall 2022\\[2mm]
\end{center}
\hrule height 1pt
}
\graphicspath{ {./images/} }
%===================================================
\NewDocumentCommand{\datesec}{o m m}{%
    \renewcommand\thesection{\DTMdate{#2}}
    \IfNoValueTF{#1}
        {\section{#3}}
        {\section[#1]{#3}}
}
%===================================================
\geometry{left=1in,right=1in,top=1in,bottom=1in}

\begin{document}
\maketitle
\tableofcontents
\section*{\datesec{2022-9-23}{}}
\paragraph*{\href{https://www.humdrum.org/guide/ch01/}{Humdrum Ch.1}}

It seems like a tool used for analyzing music similar to music21. With multiple 
funtionalities, Humdrum can be used for counting specific notes, showing certain
voice in a piece, and sometime even some specfic analysis on 
harmonic/tonal funtions. It is more like a library of tools on my idea of 
searching how similar two pieces are. We can conduct research with the same
approch, but with different methods. Starting with these simple metrics:
\begin{itemize}
    \item \textbf{Note Sequence}: How many notes are the same, how many are different?
        \begin{itemize}
            \item Might be useful for comparing two pieces in the same work, such as 
            \emph{Gymnopedie}. But will yield a very low score if the key changed.
        \end{itemize}
    \item \textbf{Note Sequence with Key}: How many notes are the same, given pieces 
    in different key but represented in a $\mathbb{Z}_7$ format.
    \begin{itemize}
        \item Can address some very basic isssues such as exact transposition,
        but not effective if the phrase changes. 
        \item Can use the technique mentioned in the proposal to
        solve the above problem. But what if the phrase was changed a lot?
    \end{itemize}
\end{itemize}
We then shall do the analysis with more and more complex methods such as counting
leading tones, and chords using current avaliable tools. In this way I believe we can come up with 
various ways to compare two pieces, but not being too impractical to do. The ultimate
goal is also simple: output a "similarity score" for two pieces.
\paragraph*{Will Read:}
\begin{itemize}
    \item Temperley, \emph{A Bayesian Approach to Key Finding}
    \item Lerdahl, \emph{Tonal Pitch Space} \textbf{Chapters *}
    \item Tymoczko, \emph{Geometry of Musical Chords} \textbf{Chapters *} 
\end{itemize}

\section*{\datesec{2022-11-4}{}}
\paragraph*{Test on modal music}
The Krumhansl-Schmuckler was tested on Bartok's \emph{14 Bagatelles Op.47 No.1}. The
algorithm regonized the right hand part as in E major/C\# minor, but ignored
the entire left hand in C Locrian/G Lydian. This is expected, since
the algorithm only outputs one key given all notes in the piece and 
it seems like notes in the left hand sums up to a less duration than
the right hand. So the KS algorithm put more weights in right hand than the
left. 
\\Link to a playable score:\url{https://musescore.com/user/4887176/scores/6403822}
\subparagraph*{Potential Solution}
We might do twice for the treble and bass clef seperately(limited to piano),
and then see if the two keys are the same. If they are, then we can say the
piece is in one key. 
\subparagraph*{Question}
How do we determine whether we should talk about keys seperately?

\paragraph*{Set-Class Similarity and Fourier Transform by Tymoczko}
Fourier transform assign two-dimensional vector whose components are:
\[V_{p,n}=(\cos(2\pi pn/12), \sin(2\pi pn/12))\]
Where for integer $n$ from $0$ to $6$ and $p$ in $\{0...11\}$ is the pitches
in a chord. Each fourier component is the sum of all such component:
\[n\text{th Fourier Component}=\sum_{p\in v}V_{p,n}\]
\begin{itemize}
    \item Voice leading and set-class similarity. Steps
    to find the minimal Euclidean voice leading between two n-note 
    multiset-classes $A$ and $B$:
        \begin{itemize}
            \item Choose a representative (prime form?) of $A$ calculate 
            the sum of its pitch classes.
            \item Find the $n$ (12/n semitones for each) 
            transpositions of $B$ with the same sum.
            \item For each of the transposition, calculate the $L_2$
            norm of $A$ and the vector. Do the same for inversions.
            \item Take the minimum of these $2n^2$ numbers and output the 
            result.
        \end{itemize}
    \item Fourier Magnitude
        \begin{itemize}
            \item In a set class space constructed by pitches of some 
            perfectly even $n$-note chord, $n\in\{1...6\}$. Note the $n$-note
            chord means the chord even seperate the $12$ tone equal 
            temperament pitches. 
            \item The $n$th Fourier component of a chord will decrease 
            as pitches move away from the subset of pitches in $n$ notes 
            chord. 
            
        \end{itemize}
\end{itemize}

\end{document}