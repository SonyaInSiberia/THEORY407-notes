\LoadClass[12pt,letterpaper]{article}
\RequirePackage{palatino}
\RequirePackage{verbatim}
\RequirePackage{amsmath,amsfonts,amsthm,amssymb,multirow,xcolor}
\RequirePackage{geometry}
\RequirePackage{graphicx}
\RequirePackage[useregional]{datetime2}
\RequirePackage{hyperref}
%===================================================
\renewcommand{\maketitle}{
\hrule height 1pt
\begin{center}
{\bf \large Meeting Notes}\\[2mm]
THEORY407, Fall 2022\\[2mm]
\end{center}
\hrule height 1pt
}
\graphicspath{ {./images/} }
%===================================================
\NewDocumentCommand{\datesec}{o m m}{%
    \renewcommand\thesection{\DTMdate{#2}}
    \IfNoValueTF{#1}
        {\section{#3}}
        {\section[#1]{#3}}
}
%===================================================
\geometry{left=1in,right=1in,top=1in,bottom=1in}

\begin{document}
\maketitle
\tableofcontents
\section*{\datesec{2022-9-23}{}}
\paragraph*{\href{https://www.humdrum.org/guide/ch01/}{Humdrum Ch.1}}

It seems like a tool used for analyzing music similar to music21. With multiple 
funtionalities, Humdrum can be used for counting specific notes, showing certain
voice in a piece, and sometime even some specfic analysis on 
harmonic/tonal funtions. It is more like a library of tools on my idea of 
searching how similar two pieces are. We can conduct research with the same
approch, but with different methods. Starting with these simple metrics:
\begin{itemize}
    \item \textbf{Note Sequence}: How many notes are the same, how many are different?
        \begin{itemize}
            \item Might be useful for comparing two pieces in the same work, such as 
            \emph{Gymnopedie}. But will yield a very low score if the key changed.
        \end{itemize}
    \item \textbf{Note Sequence with Key}: How many notes are the same, given pieces 
    in different key but represented in a $\mathbb{Z}_7$ format.
    \begin{itemize}
        \item Can address some very basic isssues such as exact transposition,
        but not effective if the phrase changes. 
        \item Can use the technique mentioned in the proposal to
        solve the above problem. But what if the phrase was changed a lot?
    \end{itemize}
\end{itemize}
We then shall do the analysis with more and more complex methods such as counting
leading tones, and chords using current avaliable tools. In this way I believe we can come up with 
various ways to compare two pieces, but not being too impractical to do. The ultimate
goal is also simple: output a "similarity score" for two pieces.
\paragraph*{Will Read:}
\begin{itemize}
    \item Temperley, \emph{A Bayesian Approach to Key Finding}
    \item Lerdahl, \emph{Tonal Pitch Space} \textbf{Chapters *}
    \item Tymoczko, \emph{Geometry of Musical Chords} \textbf{Chapters *} 
\end{itemize}

\end{document}